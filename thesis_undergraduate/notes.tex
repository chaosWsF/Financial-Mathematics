\documentclass[a4paper,12pt]{article}
\usepackage{geometry}
\usepackage{amsmath}
\usepackage{amssymb}
\usepackage{amsthm}
\usepackage{hyperref}

\geometry{a4paper, total={17cm,25cm}, left=2cm, top=2cm}
\numberwithin{equation}{section}
\hypersetup{colorlinks=true, linkcolor=blue}

\newtheorem{theorem}{Theorem}[section]
\newtheorem{corollary}{Corollary}[theorem]
\newtheorem{lemma}{Lemma}[theorem]
\theoremstyle{definition}
\newtheorem{definition}{Definition}[section]
\theoremstyle{remark}
\newtheorem{remark}{Remark}[section]

\title{
    \textit{Derivatives in Financial Markets with \\
    Stochastic Volatility} \\
    --- Notes and Extensions on Chapter 1 \& 2
}
\author{ }
\date{April 17, 2020}

\begin{document}
\maketitle
\tableofcontents\newpage

\section{The Black-Scholes Theory}
The aim of this chapter is to review the basic objects, ideas, 
and results of the classical Black-Scholes theory of derivative 
pricing. There are lots of methods for BS's theory derivation: 
\begin{enumerate}
    \item derive BS-PDE via replicating strategies 
    and self-financing property
    \item derive BS-PDE via hedging
    \item obtain pricing formula from BS-PDE by PDE methods, like 
    Feynman-Kac formula or Fourier transform.
    \item approximate price by binomial tree or Monte Carlo simulation
    \item risk-neutral valuation via martingale theory (finding ELMM)
\end{enumerate}

\subsection{Brownian Motion}
Following Samuelson, we assume that there are two assets in the 
market. One is a risk-free asset with price $\beta_t$, 
\begin{equation}
\label{eq:riskless_price}
    d \beta_{t} = r \beta_{t} dt,
\end{equation}
where $r \geq 0$ is the instantaneous interest rate for lending 
or borrowing money. Setting $\beta_{0} = 1$, we have 
$\beta_{t} = e^{rt}$ for $t \geq 0$. The other is the risky asset 
with price $X_t$, 
\begin{equation}
\label{eq:risky_price}
    d X_{t} = \mu X_{t} dt + \sigma X_{t} dW_{t},
\end{equation}
where $\mu$ is a constant mean return rate (drift rate), 
$\sigma > 0$ is a constant volatility, and $(W_{t})_{t\geq 0}$ 
is a standard Brownian motion (Wiener process).

\begin{definition}
\label{def:bm}
Brownian Motion is a real-valued stochastic process with 
continuous trajectories $t \rightarrow W_t$ that have 
independent and stationary increments, i.e. 
\begin{itemize}
    \item $W_{0} = 0$;
    \item for any $0<t_1 <\cdots< t_n$, the random variables 
    $\left(W_{t_1},W_{t_2}-W_{t_1},\dots,W_{t_n}-W_{t_{n-1}}\right)$ 
    are independent;
    \item for any $0\leq s<t$, the increment 
    $W_{t}-W_{s}\sim N(0,t-s)$. Especially, $W_{t} \sim N(0,t)$.
\end{itemize}
\end{definition}

\begin{definition}
The probability space is denoted by $(\Omega,\mathcal{F},P)$, 
$P$ is the Wiener measure.
\end{definition}

\begin{definition}
The increasing family of $\sigma$-algebras $\mathcal{F}_t$ 
generated by $(W_{s})_{s\leq t}$ are called the information 
on $W$ up to time $t$.
\end{definition}

\begin{definition}
All the sets of probability 0 in $\mathcal{F}_t$ is called 
\textit{the natural filtration} of the BM.
\end{definition}

\begin{definition}
$(X_t)_{t\geq 0}$ is \textit{adapted} to the filtration 
$(\mathcal{F}_t)_{t\geq 0}$ if $X_t$ is 
$\mathcal{F}_t$-measurable for every $t$, and $X_t$ is said to 
be $\mathcal{F}_t$-\textit{adapted}.
\end{definition}

\begin{remark}
We can summarize definition \ref{def:bm} via 
\textit{conditional characteristic functions}. For $0\leq s<t$ 
and $u \in \mathbb{R}$, 
\begin{equation}
\label{eq:bm_characteristic}
    E\left[e^{iu(W_{t}-W_{s})}|\mathcal{F}_{s}\right]
    =e^{-u^{2}(t-s)/2}.
\end{equation}
\end{remark}

\begin{remark}
The drawback of BM is that the trajectories are not of 
bounded variation, as followed: 
\begin{equation}
    E\left[\sum_{i=1}^{n}|W_{t_i}-W_{t_{i-1}}|\right]
    =nE\left[|W_{\frac{t}{n}}|\right]
    =n\sqrt{\frac{t}{n}}E\left[|W_1|\right].
\end{equation}
\end{remark}

\subsection{Stochastic Integral}
\begin{definition}
For $T$ a fixed finite time, let $(X_t)_{0\leq t\leq T}$ be a 
stochastic process adapted to $(\mathcal{F}_t)_{0\leq t\leq T}$, 
the filtration of the BM up to time $T$, such that 
\begin{equation}
\label{eq:stochastic_integrals_condition}
    E\left[\int_{0}^{T} (X_t)^{2} dt \right] < + \infty
\end{equation}
The \textit{stochastic integral} of $X_t$ with respect to BM 
is defined as a limit on $L^{2}(\Omega)$, 
\begin{equation}
\label{eq:stochastic_integral}
    \int_{0}^{t} X_s dW_s = \lim_{n\rightarrow+\infty} 
    \sum_{i=1}^{n} X_{t_{i-1}}(W_{t_i}-W_{t_{i-1}}),
\end{equation}
as the mesh size of the subdivision goes to zero.
\end{definition}

\begin{remark}
By iterated conditional expectations and the independent 
increments property of BM, we note that 
\begin{equation}
    E\left[\left(\sum_{i=1}^{n}X_{t_{i-1}}
    (W_{t_i}-W_{t_{i-1}})\right)^{2} \right]
    = E\left[\sum_{i=1}^{n}(X_{t_{i-1}})^{2}(t_{i}-t_{i-1})\right]
\end{equation}
for $t\leq T$. So, we have that 
\begin{equation}
\label{eq:stochastic_integral_expectation}
    E\left[\left( \int_{0}^{t} X_s dW_s \right)^2 \right]
    = E\left[ \int_{0}^{t} X_s^2 ds \right] < +\infty
\end{equation}
\end{remark}

\begin{remark}
Also, stochastic integral in \eqref{eq:stochastic_integral} has the 
\textit{martingale property}:
\begin{equation}
\label{eq:stochastic_integral_martingale}
    E\left[\int_{0}^{t} X_u dW_{u} | \mathcal{F}_{s}\right]
    = \int_{0}^{s} X_u dW_u \quad (a.s.\quad s\leq t).
\end{equation}
\end{remark}

\begin{definition}
The quadratic variation $\langle Y \rangle_t$ of the stochastic 
integral $Y_t = \int_{0}^{t} X_u dW_u$ is 
\begin{equation}
\label{eq:stochastic_integral_QV}
    \langle Y \rangle_t = \lim_{n\rightarrow+\infty} \sum_{i=1}^{n} (Y_{t_i} - Y_{t_{i-1}})^2
    = \int_{0}^{t} X_s^2 ds.
\end{equation}
\end{definition}

\begin{theorem}
$Y_t$ is a mean-zero, continuous and square integral martingale. 
The converse is also true: every mean-zero, continuous, and square 
integrable martingale is a Brownian stochastic integral.
\end{theorem}

\subsection{It\^o's Formula}
Based on Taylor's formula and the fact that $W_t$ is not 
differentiable, we can use 

\begin{theorem}
(It\^o's Lemma)
\begin{equation}
\label{eq:ito_BM}
    dg(W_t) = g'(W_t)dW_t + \frac{1}{2}g''(W_t)dt
\end{equation}
\end{theorem}

\begin{theorem}
(It\^o's Formula)
For $dX_t = \mu(t,X_t)dt + \sigma(t,X_t)dW_t$, 
\begin{equation}
    \label{eq:ito_general}
    \begin{split}
        dg(t,X_t) & = \frac{\partial g}{\partial t}dt+\frac{\partial g}{\partial x}dX_{t}+\frac{1}{2} \frac{\partial^{2} g}{\partial {x}^{2}}d\langle X\rangle_t \\
        & = \left(\frac{\partial g}{\partial t}+\mu(t,X_t)\frac{\partial g}{\partial x}+\frac{1}{2}\sigma^{2}(t,X_t)\frac{\partial^{2} g}{\partial {x}^{2}}\right)dt + \sigma(t,X_t)\frac{\partial g}{\partial x}dW_t ,
    \end{split}
\end{equation}
where all the partial derivatives of $g$ are evaluated at $(t,X_t)$.
\end{theorem}

\begin{corollary}
\begin{equation}
\label{eq:ito_multi}
    d(X_{t}Y_{t})=X_{t}dY_{t}+Y_{t}dX_{t}+d\langle X,Y\rangle_t,
\end{equation}
where the co-variation (aka. "bracket") of $X$ and $Y$ is given by 
\begin{equation}
    d\langle X,Y\rangle_t = \sigma_{X}(t,X_t)\sigma_{Y}(t,Y_t)dt.
\end{equation}      
\end{corollary}

\begin{remark}
    $\langle X,X\rangle_{t}=\langle X\rangle_t$.
\end{remark}

\begin{remark}
By guessing and testing, we find the solution of 
\eqref{eq:risky_price} is GBM, and the return $X_{t}/X_{0}$ is 
\textit{log-normal}, i.e.
\begin{equation}
\label{eq:GBM}
    X_{t}=
    X_{0}\exp\left[(\mu-\frac{1}{2}\sigma^{2})t+\sigma W_{t}\right].
\end{equation}
\end{remark}

\begin{remark}
Notice that bankruptcy (zero stock price) in this model is a 
permanent state, since it stays at zero for all time thereafter 
if $X_t$ becomes zero.
\end{remark}

\subsection*{Derivative Contracts}
\label{sec:derivative_contract}
\begin{itemize}
    \item Derivatives are contracts based on the underlying 
    asset price. They are also called \textit{contingent claims}.
    \item strike price $K$, \textit{maturity} date $T$
    \item \textit{Payoff} $h(X_T)$
    is the non-negative value of this contract at maturity.
    \item \textit{Derivative price} $P(t,x)$ is the value of 
    this contract at time $t$ for a underlying asset price 
    $X_t = x$.
    \item Exercise time $\tau$ is a \textit{stopping time}
    wrt the filtration $(\mathcal{F}_t)$, 
    i.e. random time s.t. the event ${\tau\leq t}$ belongs 
    to $\mathcal{F}_t$ for any $t\leq T$.
    \item "Exotic option" refers here to not a standard European 
    or American option.
\end{itemize}

\subsection{Replicating Strategies}
\begin{definition}
\textbf{(replicating)} This is a way for traders to 
find the value of an asset by replicating its cash flows or price 
movement using other assets whose values they already know.
\end{definition}

Considering a European-style derivative with payoff $h(X_{T})$, 
we replicate the derivative at maturity:
\begin{equation}
\label{eq:replicating_bs}
    a_T X_T + b_T e^{rT} = h(X_T)\quad a.s.
\end{equation}

This portfolio is to be \textit{self-financing}, i.e. 
no further funds are required after the initial investment, 
so 
\begin{equation}
\label{eq:self_financing_definition}
    d(a_{t}X_{t}+b_{t}e^{rt})=a_{t}dX_{t}+rb_{t}e^{rt}dt,
\end{equation}
which implies that 
\begin{equation}
    X_{t}da_{t}+e^{rt}db_{t}+d\langle a,X\rangle_t = 0,
\end{equation}
and integral form is 
\begin{equation}
    a_{t}X_{t}+b_{t}e^{rt}=a_{0}X_{0}+b_{0}+
    \int_{0}^{t} a_s dX_{s}+\int_{0}^{t} rb_{s}e^{rs} ds,
    \quad 0\leq t\leq T.
\end{equation}

Now we assume a pricing function $P(t,x)$ in 
Section \ref{sec:derivative_contract} exists and is regular 
enough to apply It\^o's formula \eqref{eq:ito_general}. 
\textit{No-arbitrage} assumption implies that 
\begin{equation}
\label{eq:no_arbitrage}
    a_t X_t + b_t e^{rt} = P(t,X_t)\qquad \forall\;0\leq t \leq T.
\end{equation}

Differentiating \eqref{eq:no_arbitrage} and using self-financing 
property \eqref{eq:self_financing_definition}, It\^o's formula \eqref{eq:ito_general} 
and \eqref{eq:risky_price}, we obtain 
\begin{equation}
\label{eq:BS_prototype}
    (a_{t}\mu X_{t}+b_{t}re^{rt})dt+a_{t}\sigma X_{t}dW_{t}
    =\left(\frac{\partial P}{\partial t}
    +\mu X_{t}\frac{\partial P}{\partial x} + \frac{1}{2}\sigma^{2}X_{t}^{2}\frac{\partial^{2} P}{\partial {x}^{2}}
    \right)dt + \sigma X_{t}\frac{\partial P}{\partial x}dW_{t},
\end{equation}
where all the partial derivatives of $P$ are evaluated at $(t,X_t)$. 
Equating the coefficients of the $dW_{t}$ terms gives 
\begin{equation}
\label{eq:risky_asset_weight}
    a_{t}=\frac{\partial P}{\partial x}(t,X_t).
\end{equation}
From \eqref{eq:no_arbitrage} we obtain 
\begin{equation}
\label{eq:riskless_asset_weight}
    b_{t}=(P(t,X_{t})-a_{t}X_{t})e^{-rt}.
\end{equation}
Equating the $dt$ terms in \eqref{eq:BS_prototype} gives 
\begin{equation}
\label{eq:bs_pde_0}
    r\left(P-X_{t}\frac{\partial P}{\partial x}\right)
    =\frac{\partial P}{\partial t}+\frac{1}{2}\sigma^{2}X_{t}^{2}\frac{\partial^{2} P}{\partial {x}^{2}},
\end{equation}

\begin{definition}
\textbf{(Black-Scholes partial differential equation)}
\begin{equation}
\label{eq:BS_pde}
    \mathcal{L}_{BS}(\sigma)P=0
\end{equation}
where 
\begin{equation}
\label{eq:BS_op}
    \mathcal{L}_{BS}=\frac{\partial }{\partial t}
    +\frac{1}{2}\sigma^{2}x^{2}\frac{\partial^{2} }{\partial {x}^{2}}
    +r\left(x\frac{\partial }{\partial x}-\cdot\right)
\end{equation}
holds in the domain $t\leq T$ and $x>0$.
\end{definition}

\begin{remark}
It is to be solved via the final condition $P(T,x)=h(x)$.
\end{remark}

\begin{remark}
The rate of return $\mu$ does not enter at all in the valuation 
of this portfolio.
\end{remark}

\subsection{Hedging}
There is another way to derive the Black-Scholes partial 
differential equation that emphasizes risk elimination, aka. 
\textbf{hedging}.

Let $P_{t}=P(t,X_{t})$ be the price of the option. If we sell $N_t$ 
options and hold $A_t$ risky asset, then the change in the value of 
this portfolio is $A_{t}dX_{t}-N_{t}dP_{t}$ due to being self-financing. 
This portfolio should be risk-free, so the coefficient of $dW_{t}$ 
should be zero, i.e. 
\begin{equation}
    A_{t}dX_{t}-N_{t}dP_{t}=r(A_{t}X_{t}-N_{t}P_{t})dt.
\end{equation}
Using It\^o's formula \eqref{eq:ito_general} and 
\eqref{eq:risky_price}, we have 
\begin{equation}
    \begin{split}
        A_{t}(\mu X_{t}dt+\sigma X_{t}dW_{t})-N_{t}\left\{\left(
        \frac{\partial P}{\partial t}
        +\mu X_{t}\frac{\partial P}{\partial x}
        +\frac{1}{2}\sigma^{2}X_{t}^{2}\frac{\partial^{2} P}{\partial {x}^{2}}
        \right)dt-\sigma X_{t}\frac{\partial P}{\partial x}dW_{t}\right\} \\
        =r(A_{t}X_{t}-N_{t}P_{t})dt.
    \end{split}
\end{equation}
Eliminating the $dW_{t}$ terms gives 
\begin{equation}
    A_{t}=N_{t}\frac{\partial P}{\partial x}(t,X_{t}).
\end{equation}

\begin{remark}
The derivation is known as \textit{Delta Hedging}: selling the 
option and holding a dynamically adjusted amount of the risky 
asset.
\end{remark}

\subsection{Black-Scholes Formula}
BS-PDE \eqref{eq:BS_pde} is solved with the final condition 
$h(x)=(x-K)^+$. Let $C_{BS}(t,x)$ be the price of European call option. 

\begin{theorem}
(BS formula for call options)
\begin{equation}
\label{eq:BS_call}
    C_{BS}(t,x)=xN(d_{1})-Ke^{-r(T-t)}N(d_{2}),
\end{equation}
where 
\begin{equation}
    \label{eq:BS_d1_d2}
    \begin{split}
    d_{1}&=\frac{\log(\frac{x}{K})+
    (r+\frac{1}{2}\sigma^{2})(T-t)}{\sigma\sqrt{T-t}}, \\
    d_{2}&=d_{1}-\sigma\sqrt{T-t},
    \end{split}
\end{equation}
and 
\begin{equation}
\label{eq:normal_cdf}
    N(z)=\frac{1}{\sqrt{2\pi}}\int_{-\infty}^{z} e^{-\frac{y^{2}}{2}} dy.
\end{equation}
\end{theorem}

Let $P_{BS}(t,x)$ be the price of a European put option. There is 
a \textit{model-free} relationship that follows from no-arbitrage 
arguments at time $T$.

\begin{theorem}
(put-call parity)
\begin{equation}
\label{eq:call_put_parity}
    C_{BS}(t,X_t)-P_{BS}(t,X_t)=X_{t}-Ke^{-r(T-t)},
\end{equation}
between put and call options with the same maturity and strike 
price.
\end{theorem}

\begin{theorem}
(BS formula for put option)
\begin{equation}
\label{eq:BS_put}
    P_{BS}(t,x)=Ke^{-r(T-t)}N(-d_{2})-xN(-d_{1}),
\end{equation}
where $d_{1},d_{2},N$ are as in \eqref{eq:BS_d1_d2} and 
\eqref{eq:normal_cdf}, respectively.
\end{theorem}    

\begin{remark}
Through plotting, we find that the pricing function of put option 
crosses over its terminal payoff for some (small enough) $x$, 
which does not happen with the call option function. This fact is 
important for pricing American options.
\end{remark}

\subsection{Risk-Neutral Valuation}\label{sec:risk_neutral_BS}
In the BS framework, there is a \textit{unique} probability measure 
$P^{*}$ equivalent to $P$ s.t. under this probability measure:
\begin{enumerate}
    \item the discounted price $\widetilde{X}_{t}=e^{-rt}X_{t}$ is a martingale;
    \item the expected value under $P^{*}$ of the discounted 
    payoff of a derivative gives its no-arbitrage price.
\end{enumerate}

Such a probability measure describing a \textit{risk-neutral} world 
is called an \textit{equivalent martingale measure} or risk-neutral 
measure. To find unique EMM of the discounted price, we use 
\begin{equation}
    d\widetilde{X}_{t}=\sigma\widetilde{X}_{t}
    \left[dW_{t}+\left(\frac{\mu-r}{\sigma}\right)dt\right].
\end{equation}
We set 
\begin{equation}
    \theta = \frac{\mu-r}{\sigma},
\end{equation}
called the \textit{market price of asset risk}, and define 
\begin{equation}
\label{eq:risk_neutral_BM}
    W_{t}^{*}=W_{t}+\int_{0}^{t} \theta ds = W_{t}+\theta t,
\end{equation}
so that 
\begin{equation}
\label{eq:EMM}
    d\widetilde{X}_{t}=\sigma\widetilde{X}_{t}dW_{t}^{*}.
\end{equation}

We try to construct EMM $P^{*}$, an equivalent measure to $P$, 
meaning that it has the same null sets. According to 
\textbf{Radon-Nikodym theorem}, we can change the measure. Let 
$P^{*}$ has the density $\xi_{T}^{\theta}$ wrt $P$: 
\begin{equation}
    dP^{*}=\xi_{T}^{\theta}dP
\end{equation}
where 
\begin{equation}
    \xi_{T}^{\theta}=\exp(-\theta W_{T}-\frac{1}{2}\theta^{2}T).
\end{equation}

\begin{remark}
$E[\xi_{T}^{\theta}]=1$
\end{remark}

\begin{remark}\label{Radon_Nikodym}
The \textit{Radon-Nikodym} process $(\xi_{t}^{\theta})_{0\leq t\leq T}$ 
is a martingale wrt $\mathcal{F}_t$.
\end{remark}

\begin{remark}
For any integrable random variable $Z$, 
$E^{*}[Z]=E[\xi_{T}^{\theta}Z]$.
\end{remark}

\begin{remark}
For any adapted and integrable process $(Z_t)$,\footnote{The details 
can be seen in Section 2.3 of coursenotes2003.pdf.}
\begin{equation}\label{eq:EMM_expectation}
    E^{*}[Z_{t}|\mathcal{F}_{t}]
    =\frac{1}{\xi_{s}^{\theta}}E[\xi_{t}^{\theta}Z_{t}|\mathcal{F}_s].
\end{equation}
\end{remark}

\begin{remark}\label{Girsanov_TH}
Under the probability $P^{*}$, the process $(W_{t}^{*})$ is a 
standard BM. This result in its full generality, when $\theta$ is 
an adapted stochastic process, is known as \textbf{Girsanov's theorem}.
\footnote{The details and proof can be seen in Section 4 of 
coursenotes2003.pdf.} In our case ($\theta$ is constant), it is 
easily derived by using the characterization 
\eqref{eq:bm_characteristic} and formula \eqref{eq:EMM_expectation}.
\end{remark}

\begin{definition}
\textbf{(Self-financing)} 
Let $V_t$ be the portfolio value at time $t$ 
\begin{equation}
    V_t = a_t X_t + b_t e^{rt}.
\end{equation}
Then the self-financing means that $\widetilde{V}_{t}=e^{-rt}V_{t}$ 
is a martingale under EMM $P^{*}$, i.e. 
\begin{equation}
\label{eq:value_risk_neutral}
    d\widetilde{V}_{t}=a_{t}d\widetilde{X}_{t}
    =\sigma a_{t}\widetilde{X}_{t}dW_{t}^{*}.
\end{equation}
\end{definition}

\begin{remark}
Due to being self-financing, 
\begin{equation}
\label{eq:risk_neutral_valuation}
    V_{t}=E^{*}[e^{-r(T-t)}H|\mathcal{F}_{t}]
\end{equation}
where $H=h(X_T)$.
\end{remark}

The existence of self-financing portfolio is guaranteed by 

\begin{theorem}\label{th:martingale_representation}
    (Martingale Representation Theorem) For $0\leq t\leq T$, 
    \begin{equation}
        M_{t}=E^{*}[e^{-rT}H|\mathcal{F}_{t}]
    \end{equation}
    defines a square integrable martingale under $P^{*}$ wrt 
    $(\mathcal{F}_t)$, which is also the natural filtration of the 
    BM $W^{*}$.
\end{theorem}

\begin{definition}
\textbf{(Markov Property)} 
$X_t$ is Markovian if conditioning wrt $\mathcal{F}_{t}$ is the 
same as conditioning wrt $X_t$, i.e. 
$E[\cdot|\mathcal{F}_{t}]=E[\cdot|X_{t}]$.
\end{definition}

\begin{remark}
Through this method, we can obtain the same pricing formula, i.e. 
\begin{equation}
    \begin{split}
        P(t,x)&=E^{*}[e^{-r(T-t)}h(X_{T})|X_{t}=x] \\
        &=\frac{1}{\sqrt{2\pi(T-t)}}\int_{-\infty}^{+\infty} e^{-r(T-t)}h(xe^{(r-\frac{\sigma^{2}}{2})(T-t)+\sigma z})e^{-\frac{z^{2}}{2(T-t)}} dz.
    \end{split}
\end{equation}
\end{remark}

\subsection*{American Options}
Via the theory of \textit{optimal stopping}, it can be shown 
that the price of an American derivative with payoff function $h$ 
is obtained by maximizing the expected value of the discounted 
payoff. In other words, 
\begin{equation}
\label{eq:American_formula}
    P(t,x)=\sup_{t\leq\tau\leq T} E^{*}[e^{-r(T-t)}h(X_{\tau}^{t,x})],
\end{equation}
where $(X_{\tau}^{t,x})_{\tau\geq t}$ is the stock price starting 
at $X_{t}=x$.

In order to determine the optimal stopping time $\tau^{*}$, this 
leads to a \textit{free boundary value problem}. According to a 
no-arbitrage argument, we have 

\begin{theorem}
For non-negative interest rates and no dividend paid, the price of 
an American call option is the same as its corresponding European 
option.
\end{theorem}

\begin{remark}
The price of an American put option is in general strictly higher 
than the price of the corresponding European put option.
\end{remark}

\subsection*{Complete Market}
In the previous sections we assumed the existence, uniqueness and regularity 
of the solution of the \textit{parabolic partial differential 
equation (aka. heat equation)} 
\begin{equation}
\label{eq:parabolic_pde}
    \frac{\partial u}{\partial t}
    +\frac{1}{2}\sigma^{2}(t,x)\frac{\partial^{2} u}{\partial {x}^{2}}
    +\mu(t,x)\frac{\partial u}{\partial x}-ru=0
\end{equation}
with the final condition $u(T,x)=h(x)$ in order to apply It\^o's formula. 
A sufficient condition for this is that the coefficients $\mu$ and $\sigma$ 
are regular enough and that there exists a positive constant $A$ s.t.
\begin{equation}
\label{eq:sigma_not_too_small}
    \sigma^{2}(t,x)\geq A\geq 0\quad \forall\; t\geq 0,x\in \mathcal{D}
\end{equation}
where $\mathcal{D}$ is the domain of the process $(X_t)$.

In addition, the model we have analyzed here is an example of a 
\textit{complete} market model.
\begin{definition}
    Complete market is one in which every contingent claim can be 
    replicated by a self-financing trading strategy in the stock 
    and bond.
\end{definition}

When looking at stochastic volatility market models, we shall see 
that the market is \textit{incomplete}. There is a whole family of 
EMMs, and derivatives securities cannot be perfectly hedged with 
just the stock and bond.

\newpage
\section{Stochastic Volatility Models}
The BS model needs a number of assumptions that are, to some extent, 
"counterfactual":
\begin{enumerate}
    \item continuity of the stock-price process (it does not jump),
    \item the ability to hedge continuously without transaction costs,
    \item independent Gaussian returns,
    \item constant volatility.
\end{enumerate}
We shall focus here on relaxing the last assumption by allowing 
volatility to vary randomly, for the following reason: a well-known 
discrepancy between BS-predicted European option prices and market-traded 
options prices, \textit{the smile curve}, can be accounted for by 
SV models.

\subsection{Implied Volatility and the Smile Curve}
\begin{definition}
\textbf{(Implied Volatility)}
Given an European call option with observed price $C^{obs}$, strike 
price $K$ and expiration date $T$. The implied volatility $I$ is 
the value of the volatility parameter that must go into BS formula 
\eqref{eq:BS_call} to match, 
\begin{equation}
\label{eq:implied_volatility}
    C_{BS}(t,x;K,T;I)=C^{obs}.
\end{equation}
\end{definition}

\begin{remark}
A unique non-negative IV $I>0$ can be found if 
$C^{obs}>C_{BS}(t,x;K,T;0)$ since 
\begin{equation}
    \frac{\partial C_{BS}}{\partial \sigma}
    =\frac{xe^{-\frac{d_{1}^{2}}{2}}\sqrt{T-t}}{\sqrt{2\pi}}>0.
\end{equation}
\end{remark}

\begin{remark}
The implied volatilities from put and call options of the same $K,T$ 
are the same due to put-call parity \eqref{eq:call_put_parity}.
\end{remark}

\begin{definition}
\textbf{(Smile Effect)} Implied volatilities of the market prices 
vary with strike price and the time to maturity of the contract.
\end{definition}

Before the 1987 crash, the graph of $I(K)$ for fixed $t,x,T$ 
from market options prices was often observed to be the 
$\cup$-shaped, called the smile curve or \textit{smirk}. Since 1987, 
the curve is more typically downward sloping at and near the money 
$(95\%\leq K/x\leq105\%)$ and then curves upwards for far 
out-of-the-money strikes $(K\gg x)$. Other qualitative features of 
IV from stock options are that it is higher than historical 
volatility and is often decreasing with time to maturity.

\begin{remark}
Smirk tells us that there is a premium 
charged for out-of-the-money put options and in-the-money calls 
above their BS price computed with the at-the-money IV.
\end{remark}

Bounds of $\partial I/\partial K$ can be obtained by the 
BS formula \eqref{eq:BS_call}. Since no-arbitrage assumption, call 
prices must be decreasing with $K$. Thus, differentiating 
\eqref{eq:implied_volatility} wrt $K$ gives 
\begin{equation}
    \begin{split}
        \frac{\partial C^{obs}}{\partial K}&=
        \frac{\partial C_{BS}}{\partial K}+
        \frac{\partial C_{BS}}{\partial \sigma}
        \frac{\partial I}{\partial K}\leq 0, \\
        \Rightarrow\frac{\partial I}{\partial K}&\leq 
        -\frac{\partial C_{BS}/\partial K}{\partial C_{BS}/\partial \sigma}.
    \end{split}
\end{equation}
Similarly for put options 
\begin{equation}
    \begin{split}
        \frac{\partial P^{obs}}{\partial K}&=
        \frac{\partial P_{BS}}{\partial K}+
        \frac{\partial P_{BS}}{\partial \sigma}
        \frac{\partial I}{\partial K}\geq 0, \\
        \Rightarrow\frac{\partial I}{\partial K}&\geq 
        -\frac{\partial P_{BS}/\partial K}{\partial P_{BS}/\partial \sigma}.
    \end{split}
\end{equation}

\subsection{Implied Deterministic Volatility}
One popular way to modify the lognormal model is to suppose that 
volatility is a deterministic positive function of time and stock 
price: $\sigma=\sigma(t,X_{t})$. 

In the special case $\sigma(t,x)=\sigma(t)$, we can solve the SDE
\begin{equation}
    dX_{t}=rX_{t}dt+\sigma(t)X_{t}dW_{t}^{*}
\end{equation}
under EMM $P^{*}$ by the logarithmic transformation to obtain 
\begin{equation}
    X_{T}=X_{t}\exp\left(r(T-t)-\frac{1}{2}
    \int_{t}^{T} \sigma^{2}(s) ds 
    +\int_{t}^{T} \sigma(s) dW_{s}^{*}\right),
\end{equation}
so that $\log(X_{T}/X_{t}) \sim N\left((r-\frac{1}{2}\bar{\sigma^{2}})(T-t)),(\bar{\sigma^{2}}(T-t)\right)$, 
where 
\begin{equation}
    \bar{\sigma^{2}}=\frac{1}{T-t}\int_{t}^{T} \sigma^{2}(s) ds.
\end{equation}

Thus, the answer is still the BS formula with volatility parameter 
$\sqrt{\bar{\sigma^{2}}}$, the root-mean-square volatility. In this 
time-dependent case, there is no smile across strike prices. To 
have a smile across strike prices, we need $\sigma$ to depend on $x$ 
and $t$ in this framework, which is the case of level-dependent 
volatility.

There are many competing ways --- parametric or non-parametric --- 
to estimate the volatility surface $\sigma(t,x)$ from traded option 
prices. This is called "finding the implied deterministic 
volatility." It has the advantage of preserving a complete market 
model.

\subsection{Modelling Stochastic Volatility}
In "pure" SV models, the asset price $(X_{t})_{t\geq 0}$ satisfies 
the SDE 
\begin{equation}
\label{eq:stochastic_volatility_asset_price}
    dX_{t}=\mu X_{t}dt+\sigma_{t}X_{t}dW_{t},
\end{equation}
where $(\sigma_{t})_{t\geq 0}$ is called the volatility process. 
It must satisfy some regularity conditions for the model to be 
well-defined, but \textit{it does not have to be an It\^o process: 
it can be a jump process, a Markov chain, etc}.

\begin{definition}
\textbf{(Mean Reversion)} Let us denote 
\begin{equation}
    \sigma_{t}=f(Y_{t}),
\end{equation}
where $f$ is some positive function. 
Then mean-reverting stochastic volatility means that the SDE for 
$(Y_t)$ looks like
\begin{equation}
    dY_{t}=\alpha(m-Y_{t})dt+\cdots d\hat{Z}_{t},
\end{equation}
where $(\hat{Z}_{t})_{t\geq 0}$ is a BM correlated with $(W_t)$. 
Here $\alpha$ is called the rate of mean-reversion and $m$ is the 
long-run mean level of $Y_t$.
\end{definition}

\begin{definition}
\textbf{(Ornstein-Uhlenbeck process)} A stochastic process 
satisfies the SDE 
\begin{equation}
\label{eq:OU_process}
    dY_{t}=\alpha(m-Y_{t})dt+\beta d\hat{Z}_{t}.
\end{equation}
\end{definition}

\begin{remark}
OU process is a Gaussian process explicitly given in terms of its 
starting value $y$ by 
\begin{equation}
\label{eq:OU_solution}
    Y_{t}=m+(y-m)e^{-\alpha t}+\beta\int_{0}^{t} e^{-\alpha(t-s)} d\hat{Z}_{s},
\end{equation}
so that $Y_{t}\sim N(m+(y-m)e^{-\alpha t},\frac{\beta^{2}}{2\alpha}(1-e^{-2\alpha t}))$. 
Its \textit{invariant distribution}, obtained as 
$t\rightarrow\infty$, is $N(m,\frac{\beta}{2\alpha})$, which does 
not depend on $y$.
\end{remark}

\begin{remark}
The second BM $(\hat{Z}_{t})$ is typically correlated with the BM 
$(W_{t})$ driving \eqref{eq:stochastic_volatility_asset_price}. We 
denote by $\rho\in[-1,1]$ the instantaneous correlation coefficient 
defined by $d\langle W,\hat{Z}\rangle_{t}=\rho dt$. It is also 
convenient to write 
\begin{equation}
\label{eq:bm_relationship}
    \hat{Z}_{t}=\rho W_{t}+\sqrt{1-\rho^{2}}Z_{t},
\end{equation}
where $(Z_{t})$ is a standard BM independent of $(W_{t})$.
\end{remark}
Some facts and assumptions are: 
\begin{itemize}
    \item It is often found from financial data that $\rho<0$, and there are 
    economic arguments for a negative correlation or \textit{leverage effect} 
    between stock price and volatility shocks.
    \item Notice the fat tails of stock-price distribution due to the 
    random volatility. In particular, the negative correlation causes 
    the tails to be asymmetric: the left tail is fatter.
    \item We assume $\rho$ a constant from now on for simplicity 
    and because it is taken to be such in most practical situations.
    \item We still consider the underlying probability space 
    $(\Omega,\mathcal{F},P)$, where now we can take 
    $\Omega=C([0,\infty):\mathbb{R}^{2})$, the space of all continuous 
    trajectories $(W_{t}(\omega),Z_{t}(\omega))=\omega(t)$ in $\mathbb{R}^{2}$.
    \item The filtration $(\mathcal{F}_{t})_{t\geq 0}$ represents the information 
    on the two BMs.
\end{itemize}
Some common driving processes $(Y_{t})$ are: 
\begin{enumerate}
    \item Log-Normal (LN) $dY_{t}=c_{1}Y_{t}dt+c_{2}Y_{t}d\hat{Z}_{t}$
    \item Ornstein-Uhlenbeck (OU) $dY_{t}=\alpha(m-Y_{t})dt+\beta d\hat{Z}_{t}$
    \item Feller or Cox-Ingersoll-Ross (CIR) $dY_{t}=\kappa(m'-Y_{t})dt+\nu\sqrt{Y_{t}}d\hat{Z}_{t}$
\end{enumerate}
Some models often studied in the literature are listed in Table 
\ref{tab:sv_models}.

\begin{table}[h]
    \centering
    \caption{Models of Stochastic Volatility}
    \begin{tabular}{cccc}
        \hline\hline
        Authors & Correlation & $f(y)$ & Process \\
        \hline
        Hull-White & $\rho=0$ & $\sqrt{y}$ & LN \\
        Scott & $\rho=0$ & $\exp(y)$ & OU \\
        Stein-Stein & $\rho=0$ & $|y|$ & OU \\
        Ball-Roma & $\rho=0$ & $\sqrt{y}$ & CIR \\
        Heston & $\rho\neq 0$ & $\sqrt{y}$ & CIR \\
        \hline\hline
    \end{tabular}
    \label{tab:sv_models}
\end{table}

\subsection{Derivative Pricing}\label{sec:sv_BS}
When the volatility is a \textit{Markovian It\^o process}, we can 
find a pricing function for European derivatives of the form 
$P(t,X_{t},Y_{t})$ from no-arbitrage arguments, as in the BS case. 
We will derive the pricing PDE.\footnote{This PDE is not solved here.} 
Assuming that volatility is a function of a mean-reverting OU 
process: 
\begin{equation}\label{eq:SV_SDE_0}
    \begin{split}
        dX_{t}&=\mu X_{t}dt+\sigma_{t} X_{t}dW_{t}, \\
        \sigma_{t}&=f(Y_{t}), \\
        dY_{t}&=\alpha(m-Y_{t})dt+\beta d\hat{Z}_{t}, \\
        \hat{Z}_{t}&=\rho W_{t}+\sqrt{1-\rho^{2}}Z_{t},
    \end{split}
\end{equation}
where $W_{t}$ and $Z_{t}$ are independent BMs.

Unlike the BS case, it is not sufficient to hedge solely with the 
underlying asset, since there are two Wiener processes. Thus, we 
try to hedge with the underlying asset and \textit{another option} 
has a different expiration date. Let $P^{(1)}$ be the price of a 
European derivative with expiration date $T_1$ and payoff function 
$h$, $P^{(2)}$ be the price of a European contract with 
$T_{2}>T_{1}>t$ and $h$. We try to find processes 
$\{a_{t},b_{t},c_{t}\}$ s.t. 
\begin{equation}
\label{eq:hedging_eq_sv_0}
    P^{(1)}(T_{1},X_{T_{1}},Y_{T_{1}})=a_{T_{1}}X_{T_{1}}
    +b_{T_{1}}\beta_{T_{1}}+c_{T_{1}}P^{(2)}(T_{1},X_{T_{1}},Y_{T_{1}}),
\end{equation}
where $\beta_{t}=e^{rt}$ is the price of a risk-free bond under 
the prevailing short-term constant interest rate $r$. Also, the 
portfolio is to be self-financing, so that 
\begin{equation}
\label{eq:self_financing_sv}
    dP^{(1)}=a_{t}dX_{t}+b_{t}re^{rt}dt+c_{t}dP^{(2)}.
\end{equation}

If such a portfolio can be found then, in order for there to be 
no-arbitrage opportunities, it must be that 
\begin{equation}
\label{eq:hedging_eq_sv}
    P^{(1)}(t,X_{t},Y_{t})=a_{t}X_{t}+b_{t}\beta_{t}
    +c_{t}P^{(2)}(t,X_{t},Y_{t}),\quad \forall\;t<T_{1}.
\end{equation}

\begin{theorem}
(Two-Dimensional Version of It\^o's Formula)
\begin{equation}
\label{eq:ito_formula_2dim}
    dg(t,X_{t},Y_{t})=\frac{\partial g}{\partial t}dt+\frac{\partial g}{\partial x}dX_{t}
    +\frac{\partial g}{\partial y}dY_{t}+\frac{1}{2}
    \left(\frac{\partial^{2} g}{\partial {x}^{2}}d\langle X\rangle_{t}
    +2\frac{\partial^{2} g}{\partial x\partial y}d\langle X,Y\rangle_{t}
    +\frac{\partial^{2} g}{\partial {y}^{2}}d\langle Y\rangle_{t}\right).
\end{equation}
\end{theorem}

Applying this to both sides of \eqref{eq:self_financing_sv} yields 
\begin{equation}
    \label{eq:ou_sv_0}
    \begin{split}
        &\left(\frac{\partial P^{(1)}}{\partial t}+\mathcal{M}_{1}P^{(1)}\right)dt+
        \frac{\partial P^{(1)}}{\partial x}dX_{t}+\frac{\partial P^{(1)}}{\partial y}dY_{t} \\
        &=\left(a_{t}+c_{t}\frac{\partial P^{(2)}}{\partial x}\right)dX_{t}
        +c_{t}\frac{\partial P^{(2)}}{\partial y}dY_{t}
        +\left[c_{t}\left(\frac{\partial }{\partial t}+\mathcal{M}_{1}\right)P^{(2)}+b_{t}re^{rt}\right]dt,
    \end{split}
\end{equation}
where 
\begin{equation}
    \mathcal{M}_{1}=\frac{1}{2}f(y)^{2}x^{2}\frac{\partial^{2} }{\partial {x}^{2}}
    +\rho\beta xf(y)\frac{\partial^{2} }{\partial x\partial y}
    +\frac{1}{2}\beta^{2}\frac{\partial^{2} }{\partial {y}^{2}}.
\end{equation}
Equating $dY_{t}, dX_{t}$ and using \eqref{eq:hedging_eq_sv}, 
which gives 
\begin{equation}
    \begin{split}
        c_{t}&=\frac{\partial P^{(1)}/\partial y}{\partial P^{(2)}/\partial y}, \\
        a_{t}&=\frac{\partial P^{(1)}}{\partial x}-c_{t}\frac{\partial P^{(2)}}{\partial x}, \\
        b_{t}&=(P_{t}^{(1)}-a_{t}X_{t}-c_{t}P^{(2)})e^{-rt}.
    \end{split}
\end{equation}
Comparing $dt$ terms in \eqref{eq:ou_sv_0} gives 
\begin{equation}\label{eq:eliminate_dt_sv}
    \left(\frac{\partial P^{(1)}}{\partial y}\right)^{-1}
    \mathcal{M}_{2}P^{(1)}(t,X_{t},Y_{t})
    =\left(\frac{\partial P^{(2)}}{\partial y}\right)^{-1}
    \mathcal{M}_{2}P^{(2)}(t,X_{t},Y_{t}),
\end{equation}
where 
\begin{equation}
    \mathcal{M}_{2}=\frac{\partial }{\partial t}+\mathcal{M}_{1}
    +r\left(x\frac{\partial }{\partial x}-\cdot\right).
\end{equation}

Now, the left-hand side of \eqref{eq:eliminate_dt_sv} contains 
terms depending on $T_{1}$ but not $T_{2}$ and vice versa for the 
right-hand side. Thus, both sides must be equal to a function that 
does not depend on expiration date $T_{1},\,T_{2}$. We denote this 
function by 
\begin{equation}
    -\left(\alpha(m-y)-\beta\left(\rho\frac{\mu-r}{f(y)}
    +\gamma(t,x,y)\sqrt{1-\rho^{2}}\right)\right)
\end{equation}
where $\gamma(t,x,y)$ is an arbitrary function.\footnote{The 
original idea was to guess the form.} Thus, the pricing function 
$P(t,x,y)$ must satisfy

\begin{definition}
\textbf{(Kolmogorov or Feynman-Kac PDE)}
\begin{equation}
\label{eq:sv_pde}
    \begin{split}
        \frac{\partial P}{\partial t}+\frac{1}{2}f(y)^{2}x^{2}\frac{\partial^{2} P}{\partial {x}^{2}}
        +\rho\beta xf(y)\frac{\partial^{2} P}{\partial x\partial y}
        +\frac{1}{2}\beta^{2}\frac{\partial^{2} P}{\partial {y}^{2}} \\
        +r\left(x\frac{\partial P}{\partial x}-P\right)
        +[(\alpha(m-y)-\beta\Lambda(t,x,y))]\frac{\partial P}{\partial y}=0
    \end{split}
\end{equation}
where 
\begin{equation}
\label{eq:sv_pde_Lambda}
    \Lambda(t,x,y)=\rho\frac{\mu-r}{f(y)}+\gamma(t,x,y)\sqrt{1-\rho^{2}}.
\end{equation}
The terminal condition is $P(T,x,y)=h(x)$.
\end{definition}

\begin{remark}
The function $\gamma$ in \eqref{eq:sv_pde_Lambda} is the 
\textit{risk premium factor} from the second source of randomness 
$Z_{t}$ in \eqref{eq:bm_relationship} that drives the volatility. 
In the perfectly correlated case $|\rho|=1$ it does not appear, 
as expected. The reason for this terminology is the calculation 
\begin{equation}
    \begin{split}
        dP(t,X_{t},Y_{t})=&\left[\underline{\frac{\mu-r}{f(y)}}
        \left(
        xf(y)\frac{\partial P}{\partial x}
        +\underline{\beta\rho\frac{\partial P}{\partial y}}
        \right)+rP+\underline{\gamma\beta
        \sqrt{1-\rho^{2}}\frac{\partial P}{\partial y}}
        \right]dt \\
        &+\left(xf(y)\frac{\partial P}{\partial x}
        +\beta\rho\frac{\partial P}{\partial y}\right)dW_{t}
        +\beta\sqrt{1-\rho^{2}}\frac{\partial P}{\partial y}dZ_{t},
    \end{split}
\end{equation}
which is obtained by the It\^o's formula \eqref{eq:ito_formula_2dim} 
and the PDE \eqref{eq:sv_pde} satisfied by $P$. From this expression 
we see that an infinitesimal fractional increase in the volatility 
risk $\beta$ increases the infinitesimal rate of return on the option 
by $\gamma$ times that fraction, in addition to the increase from 
the excess return-to-risk ratio $(\mu-r)/f(y)$.
\end{remark}

\subsection{Pricing with Equivalent Martingale Measures}
\label{sec:sv_emm}

We give an alternative derivation of the no-arbitrage derivative 
price using \textit{risk-neutral} theory, again for the model 
\eqref{eq:SV_SDE_0}, but the procedure is valid for 
\textit{general models}, including non-Markovian ones.

Suppose that there is a an EMM $P^{*}$ under which the discounted 
stock price $\widetilde{X}_{t}=e^{-rt}X_{t}$ is a martingale. Thus, 
we can price any derivative with expiration $T$ and payoff $h$ by 
\begin{equation}
    V_{t}=E^{*}[e^{-r(T-t)}H|\mathcal{F}_{t}]\quad \forall\;t\leq T,
\end{equation}
where $H=h(X_{T})$, then there is no-arbitrage opportunity (See 
Section \ref{sec:risk_neutral_BS}).

We absorb the drift term of $\widetilde{X}_{t}$ in its martingale 
term by setting 
\begin{equation}
    W_{t}^{*}=W_{t}+\int_{0}^{t} \frac{\mu-r}{f(Y_{s})} ds.
\end{equation}

Any shift of the \textit{second independent BM} of the form 
\begin{equation}
    Z_{t}^{*}=Z_{t}+\int_{0}^{t} \gamma_{s} ds
\end{equation}
will not change the drift of $\widehat{X}_{t}$. Here $(\gamma_{t})$ 
is any adapted (and suitably regular) process. By \textit{
Girsanov's theorem}\footnote{See Remark \ref{Girsanov_TH}.}, 
$(W^{*})$ and $(Z^{*})$ are independent standard BMs under a 
measure $P^{*(\gamma)}$ defined by\footnote{The first equation is 
Radon-Nikodym derivative mentioned in Remark \ref{Radon_Nikodym}.}
\begin{equation}
    \begin{split}
        \frac{d P^{*(\gamma)}}{d P}&=\exp
        \left\{-\int_{0}^{T} \theta_{t}^{(1)} dW_{t}
        -\int_{0}^{T} \theta_{t}^{(2)} dZ_{t}
        -\frac{1}{2}
        \int_{0}^{T} [(\theta_{t}^{(1)})^{2}+
        (\theta_{t}^{(2)})^{2}] dt\right\}, \\
        \theta_{t}^{(1)}&=\frac{\mu-r}{f(Y_{t})},\quad
        \theta_{t}^{(2)}=\gamma_{t}.
    \end{split}
\end{equation}

Technically, we shall make an assumption on the pair 
$(\theta_{t}^{(1)},\theta_{t}^{(2)})$ so that $P^{*(\gamma)}$ is 
well-defined as a probability measure.\footnote{For Example, Heston 
model needs the Feller condition for the existence of ELMM, detailed 
in Section 3 of Wong2006.pdf and arxiv20191230.pdf.} In particular, 
this will be the case if $f$ is bounded away from zero and 
$(\gamma_{t})$ is bounded. Then, under $P^{*(\gamma)}$, the SDEs 
\eqref{eq:SV_SDE_0} become 
\begin{equation}
    \begin{split}
        dX_{t}&=rX_{t}dt+f(Y_{t})X_{t}dW_{t}^{*}, \\
        dY_{t}&=\left[\alpha(m-Y_{t})-\beta\left(\rho
        \frac{\mu-r}{f(Y_{t})}+\gamma_{t}\sqrt{1-\rho^{2}}
        \right)\right]dt+\beta d\hat{Z}_{t}^{*}, \\
        \hat{Z}_{t}^{*}&=\rho W_{t}^{*}+\sqrt{1-\rho^{2}}Z_{t}^{*}.
    \end{split}
\end{equation}

Any allowable choice of $\gamma$ leads to an EMM $P^{*(\gamma)}$ 
and the possible no-arbitrage prices are\footnote{This expectation can 
be obtained by \eqref{eq:EMM_expectation}.}
\begin{equation}\label{eq:sv_risk_neutral_valuation}
    V_{t}=E^{*(\gamma)}[e^{-r(T-t)}H|\mathcal{F}_{t}].
\end{equation}

The process $(\gamma_{t})$ is called the risk premium factor or the 
\textit{market price of volatility risk} from the second source of 
randomness $Z$ that drives the volatility. As a result, the 
\textit{martingale representation theorem} 
\ref{th:martingale_representation} says that under $P^{*(\gamma)}$ 
martingale $M_{t}=e^{-rt}V_{t}$ is a stochastic integral 
wrt $(W^{*},Z^{*})$:
\begin{equation}\label{eq:martingale_representation_sv}
    M_{t}=M_{0}+\int_{0}^{t} \eta_{s}^{(1)} dW_{s}^{*}
    +\int_{0}^{t} \eta_s^{(2)} dZ^{*}_{s}
\end{equation}
for some adapted and suitably bounded processes $(\eta_{t}^{(1)})$ 
and $(\eta_{t}^{(2)})$. So, we cannot replicate the claim by 
trading in stock and bond only due to the last integral in 
\eqref{eq:martingale_representation_sv}.\footnote{Q: How about VIX 
options? If we trade volatility derivatives, is there any difference?} 
We can, however, hedge one derivative contract $P^{(1)}$ with the 
stock and another derivative security $P^{(2)}$. The calculation 
is like the Markovian case in Section \ref{sec:sv_BS}, but the 
hedging ratios are \textit{non-unique} since they depend on 
$\gamma$. This procedure is usually unsatisfactory owing to the 
\textit{higher transaction costs} and \textit{less liquidity} 
associated with trading the second derivative.

\subsection{Implied Volatility as a Function of Moneyness}
Another reason that IV is a particularly useful measure of the 
performance of a SV model is that IV is a function of a European 
option contract's \textit{moneyness} $K/x$. Given any Markovian SV model under 
which the stock price satisfies 
\begin{equation}\label{eq:stock_sde_markovian_sv}
    dX_{s}=rX_{s}ds+\sigma_{s}X_{s}dW_{s}^{*}
\end{equation}
under some risk-neutral measure $P^{*(\gamma)}$, suppose it is now 
time $t,X_{t}=x$ and define $\widetilde{X}=X/x$. Then 
$(\widetilde{X}_{s})_{s\geq t}$ satisfies the same SDE 
\eqref{eq:stock_sde_markovian_sv}, with initial value 
$\widetilde{X}_{t}=1$, neither of which depends on $x$. The call 
option price 
\begin{equation}
    \begin{split}
        C&=E^{*(\gamma)}[e^{-r(T-t)}(X_{T}-K)^{+}|X_{t}=x,
        \sigma_{t}] \\
        &=E^{*(\gamma)}[e^{-r(T-t)}(x\widetilde{X}_{T}-K)^{+}
        |\widetilde{X}_{t}=1,\sigma_{t}] \\
        &=KE^{*(\gamma)}[e^{-r(T-t)}(\frac{x}{K}\widetilde{X}_{T}
        -1)^{+}|\widetilde{X}_{t}=1,\sigma_{t}] \\
        &=KQ_{1}(t,K/x;T)
    \end{split}
\end{equation}
for some function $Q_{1}$ depending on $K/x$ but not on $x$ and 
$K$ separately. The IV $I$ is computed from 
\begin{equation}
    C=C_{BS}(t,x;K,I),
\end{equation}
and from the BS-formula \eqref{eq:BS_call} we have 
\begin{equation}
    C_{BS}(t,x;K,I)=KQ_{2}(t,K/x;I)
\end{equation}
for some function $Q_{2}$ also depending on $K/x$ but not on $x$ 
and $K$ separately. From the relation $KQ_{1}(t,K/x;T)=KQ_{2}(t,K/x;I)$ 
we see that $I$ must be a function of moneyness $K/x$ but not of 
$K$ and $x$ separately.

This is useful because it tells us that we can obtain the implied 
volatility curve predicted by a SV model by solving the PDE 
\eqref{eq:sv_pde} with terminal condition $h(x)=(x-K)^{+}$ for 
\textit{a fixed strike price} $K$. This is because plotting the 
resulting implied volatilities as a function of moneyness for 
different starting values $x$ gives the same curve as if we were 
varying $K$.

\subsection{Market Price of Volatility Risk and Data}
Now, there are more details.

\textbf{Bound:} In Section \ref{sec:sv_emm}, we know that 
\eqref{eq:sv_risk_neutral_valuation} is a possible no-arbitrage 
derivative pricing formula for any EMM $P^{*(\gamma)}$. If the 
volatility is unbounded, then the range of European call option 
prices given by \eqref{eq:sv_risk_neutral_valuation} with 
$H=(X_{T}-K)^{+}$ is between the price of stock and intrinsic value 
of the contract, $(X_{T}-K)^{+}\leq V_{t}\leq X_{t},$ and the 
extreme values are attained for some EMMs. When volatility is assumed 
to be bounded with $\sigma_{t}\in[\sigma_{\mathrm{min}},\sigma_{\mathrm{max}}]$, 
the bounds are $C_{BS}(\sigma_{\mathrm{min}})\leq V_{t}\leq C_{BS}(\sigma_{\mathrm{max}})$. 

\textbf{Super-hedge:} The viewpoint, market selects a \textit{unique} EMM 
under which derivative contracts are priced, is sometimes called 
\textit{selecting an approximating complete market}. Although discounted 
derivative prices cannot be replicated by stock and bond alone, 
they can be \textit{super-replicated}: for example, buying the 
stock at time $t<T$ and holding it until expiration 
\textit{super-hedges} a short call position because 
$X_{T}>(X_{T}-K)^{+}$; this may yield a profit but never a loss. 
This (trivial) strategy is very expensive (it costs \$ $X_t$), so 
many researches concern finding the cheapest super-hedging strategy.

\textbf{Parameter Estimation:} When we estimate the parameters for 
our model, we could use econometric methods such as maximum 
likelihood or method of moments on historical stock-price 
time-series data to find $(\alpha,\beta,m,\rho)$ plus the present 
volatility in the model \eqref{eq:SV_SDE_0}. Then we would 
need some derivative data to estimate $\gamma$, assuming for 
instance that it is constant. The common practice, called 
\textit{cross-sectional fitting}, is to estimate all the 
parameters from derivative data, usually \textit{at-the-money} 
European option prices (or a section of the observed implied 
volatility surface). This ignores the statistical basis for the 
modeling but is easier to implement than time-series methods, 
which suffer because the $(\sigma_{t})$ process is not directly 
observable. If today is time $t=0$ and we denote 
$\vartheta=(\alpha,\beta,m,\rho,\sigma_{0},\gamma,\mu)$, 
then a typical \textit{least-square fit} is to observe call option 
prices $C^{obs}(K,T)$ for $(K,T)$ in some set $\mathcal{K}$ and to 
solve 
\begin{equation}
    \min_{\vartheta} \sum_{(K,T)\in\mathcal{K}} (C(K,T;\vartheta)
    -C^{obs}(K,T))^{2},
\end{equation}
where $C(K,T;\vartheta)$ is the model-predicted call option price 
(either from solving the PDE \eqref{eq:sv_pde} with 
$h(x)=(x-K)^{+}$ or from Monte Carlo simulation). But this process 
can be very slow and computationally intensive.

\newpage
\section{Heston Stochastic Volatility Model}\label{sec:Heston}
This section introduces the \textit{Heston model} which is used by 
Steven L. Heston in 1993.\footnote{All researches mentioned in this 
section can be found in the reference of Heston1993\_RFS.pdf.} 
Heston used a new technique to derive a closed-form solution for 
the price of European call option on an asset with SV. The model 
allows arbitrary correlation between volatility and spot-asset 
returns. He also shows that correlation is important for explaining 
return skewness and strike-price biases in the BS model. The 
solution technique is based on \textit{characteristic functions}.

\subsection{Stochastic Differential Equations}
Assuming that the spot asset price $S_t$ satisfies the diffusion 
equation, the volatility follows the OU process, and It\^o's lemma 
\eqref{eq:ito_BM} shows that the variance $v_{t}$ follows the CIR 
process, 
\begin{equation}
\label{eq:Heston_SDE}
\begin{split}
    dS_{t}&=\mu S_{t}dt+\sqrt{v_{t}}S_{t}dW_{t}, \\
    d\sqrt{v_{t}}&=-\beta\sqrt{v_{t}}dt+\delta d\hat{Z}_{t}, \\
    dv_{t}&=\kappa(\theta-v_{t})dt+\sigma\sqrt{v_{t}}d\hat{Z}_{t}, \\
    d\langle W,\hat{Z}\rangle_{t}&=\rho dt.
\end{split}
\end{equation}
where $W_{t},\hat{Z}_{t}$ are Wiener processes. Assume $\kappa,\theta,\sigma>0$ 
and $-1<\rho<1$. Also, we assume the variance process satisfies 
the Feller condition $2\kappa\theta>\sigma^{2}$, so it is always 
positive and cannot reach zero.

For simplicity, we assume a constant interest rate $r$. Therefore, 
the price at time $t$ of a unit discount bond that matures at time 
$t+\tau$ is 
\begin{equation}
\label{eq:Heston_riskless}
    P(t,t+\tau)=e^{-r\tau}.
\end{equation}

\subsection{Heston Partial Differential Equation}
Standard arbitrage arguments demonstrate that the value of any 
asset $U(S,v,t)$ must satisfy the PDE\footnote{This is Feynman-Kac 
PDE derived in Section \ref{sec:sv_BS}.}
\begin{equation}
\label{eq:sv_pde_Heston}
\begin{split}
    \frac{1}{2}vS^{2} & \frac{\partial^{2} U}{\partial {S}^{2}}
    +\rho\sigma vS\frac{\partial^{2} U}{\partial {S} \partial {v}}
    +\frac{1}{2}\sigma^{2}v\frac{\partial^{2} U}{\partial {v}^{2}}+rS\frac{\partial U}{\partial S} \\
    & +[\kappa(\theta-v_{t})-\lambda(S,v,t)]\frac{\partial U}{\partial v}
    -rU+\frac{\partial U}{\partial t}=0.
\end{split}    
\end{equation}

The unspecified term $\lambda(S,v,t)$ represents \textit{the price 
of volatility risk}, and must be independent of the particular 
asset. To motivate the choice of $\lambda(S,v,t)$, Heston noted 
that in Breeden's (1979) consumption-based model, 
\begin{equation}
\label{eq:volatility_risk_price}
    \lambda(S,v,t)dt=\gamma\,\mathrm{Cov}[dv,dC/C],
\end{equation}
where $C_{t}$ is the consumption rate and $\gamma$ is the 
relative-risk aversion of an investor. Consider the consumption 
process that emerges in the Cox, Ingersoll and Ross (1985) model 
\begin{equation}
\label{eq:CIR_consumption}
    dC_{t}=\mu_{c}v_{t}C_{t}dt+\sigma_{c}\sqrt{v_{t}}C_{t}d\hat{Z}_{c,t},
\end{equation}
where $\hat{Z}_{c,t}$ is another Wiener process with the constant 
correlations with $W_{t},\hat{Z}_{t}$. This generates a risk 
premium proportional to $v$, i.e. $\lambda(S,v,t)=\lambda v$.

A European call option with strike price $K$ and maturing at time 
$T$ satisfies the PDE \eqref{eq:sv_pde_Heston} subject to the 
following boundary conditions:
\begin{equation}
    \label{eq:Heston_boundary_conditions}
    \begin{split}
        U(S,v,T)&=(S_{T}-K)^{+}, \\
        U(0,v,t)&=0, \\
        \frac{\partial U}{\partial S}(\infty,v,t)&=1, \\
        \left(rS\frac{\partial U}{\partial S}+\kappa\theta
        \frac{\partial U}{\partial v}-rU
        +\frac{\partial U}{\partial t}\right)(S,0,t)&=0, \\
        U(S,\infty,t)&=S_{t}.
    \end{split}
\end{equation}

\begin{remark}
Although we will use this form of the risk premium, the pricing 
results are obtained by arbitrage and do not depend on the other 
assumptions of the Breeden (1979) or CIR (1985) models.
\end{remark}

\subsection{The Call Option Price}
By analogy with the BS formula, we guess a solution of the form
\footnote{This is a corollary of Girsanov's Theorem.}
\begin{equation}
\label{eq:European_call_Heston}
    C(S,v,t)=S_{t}\,P_{1}-KP(t,T)\,P_{2},
\end{equation}
where the first term is the present value of the spot asset upon 
optimal exercise, and the second term is the present value of the 
strike-price payment. Both of these terms must satisfy the 
original PDE \eqref{eq:sv_pde_Heston}. For notional convenience, 
let
\begin{equation}
\label{eq:stock_log_Heston}
    x_{t}=\log(S_{t}).
\end{equation}
Substituting the proposed solution \eqref{eq:European_call_Heston} 
into the original PDE \eqref{eq:sv_pde_Heston} shows that $P_1$ and 
$P_2$ must satisfy the PDEs\footnote{Just set $S_{t}=1,\,K=0$ and 
$S_{t}=0,\,K=-1/P(t,T)$.}
\begin{equation}\label{eq:heston_factors}
    \begin{split}
        \frac{1}{2}v & \frac{\partial^{2} P_{j}}{\partial {x}^{2}}
        +\rho\sigma v\frac{\partial^{2} P_{j}}{\partial {x} \partial {v}}
        +\frac{1}{2}\sigma^{2}v\frac{\partial^{2} P_{j}}{\partial {v}^{2}}
        +(r+u_{j}v)\frac{\partial P_{j}}{\partial x} \\
        & +(a-b_{j}v)\frac{\partial P_{j}}{\partial v}
        +\frac{\partial P_{j}}{\partial t}=0.
    \end{split}    
\end{equation}
for $j=1,2$, where 
\begin{equation}
    u_{1}=\frac{1}{2},\quad u_{2}=-\frac{1}{2},\quad 
    a=\kappa\theta,\quad b_{1}=\kappa+\lambda-\rho\sigma,\quad 
    b_{2}=\kappa+\lambda.
\end{equation}
For the option price to satisfy the terminal conditions in 
\eqref{eq:Heston_boundary_conditions}, these PDEs 
\eqref{eq:heston_factors} are subject to the terminal condition
\begin{equation}\label{eq:heston_discounters}
    P_{j}(x,v,T;K)=1_{\{x_{T}\geq\ln{K}\}}.
\end{equation}

\subsection{Risk-Neutral Valuation}\label{sec:heston_Riccati}
Following Heston, we know that $P_{1,2}$ are risk-neutral when 
$x_{t},v_{t}$ satisfy respectively\footnote{In fact, we will 
choose $j=2$ and set $\lambda=0$.}
\begin{equation}
    \label{eq:heston_EMM}
    \begin{split}
        dx_{t}&=(r+u_{j}v_{t})dt+\sqrt{v_{t}}dW^{*}_{t}, \\
        dv_{t}&=(a-b_{j}v_{t})dt+\sigma\sqrt{v_{t}}d\hat{Z}^{*}_{t}.
    \end{split}
\end{equation}
Then we obtain 
\begin{equation}
    P_{j}(x,v,t;K)=E^{*}[1_{\{x_{T}\geq\ln{K}\}}|x_{t}=x,v_{t}=v].
\end{equation}

They are not immediately available in closed form. However, we know 
that\footnote{There are details in the Appendix of Heston1993\_RFS.pdf.} 
their characteristic functions satisfy the same PDEs 
\eqref{eq:heston_factors}, subject to the terminal condition 
under the risk-neutral measure
\begin{equation}
    \Phi_{j}(x,v,T;u)=e^{iux}.
\end{equation}

To obtain the solution, we guess the form of characteristic functions
\footnote{This exploits the linearity of the coefficients in the 
PDE \eqref{eq:heston_factors}.}
\begin{equation}
    \Phi_{j}(\tau,u)=E^{*}\left[e^{iux_{T}}|x_{t}=x,v_{t}=v\right]
    =\exp[{C_{j}(\tau,u)+D_{j}(\tau,u)v+iux}],
\end{equation}
where $\tau=T-t$. Substituting these into \eqref{eq:heston_factors} 
shows a \textit{Riccati equation} and a straightforward ODE
\begin{equation}
    \begin{split}
        \frac{dD_j}{d\tau}&=\rho\sigma iuD_{j}-\frac{1}{2}u^{2}
        +\frac{1}{2}\sigma^{2}D_{j}^{2}+u_{j}iu-b_{j}D_{j}, \\
        \frac{dC_j}{d\tau}&=riu+aD_{j}.
    \end{split}
\end{equation}
Heston specifies the initial conditions $C_{j}(0,u)=D_{j}(0,u)=0$. 
Then the results are 
\begin{equation}
    \begin{split}
        D_{j}(\tau,u)&=\frac{b_{j}-\rho\sigma iu+d_{j}}{\sigma^{2}}
        \left(\frac{1-e^{d_{j}\tau}}{1-g_{j}e^{d_{j}\tau}}\right), \\
        C_{j}(\tau,u)&=riu\tau+\frac{a}{\sigma^{2}}\left[(b_{j}-
        \rho\sigma iu+d_{j})\tau-2\ln\left(\frac{1-g_{j}e^{d_{j}\tau}}{1-g_{j}}
        \right)\right],
    \end{split}
\end{equation}
where 
\begin{equation}
    \begin{split}
        d_{j}&=\sqrt{(\rho\sigma iu-b_{j})^{2}-\sigma^{2}(2u_{j}iu-u^{2})}, \\
        g_{j}&=\frac{b_{j}-\rho\sigma iu+d_{j}}{b_{j}-\rho\sigma iu-d_{j}}.
    \end{split}
\end{equation}

\subsection{The Heston Pricing Formula}
Through some calculations, we find that 
\begin{equation}
    P(t,T)\Phi_{2}(\tau,u)=e^{-r(T-t)}\Phi_{2}(\tau,u)
    =e^{x}\Phi_{1}(\tau,u+i)=S_{t}\Phi_{1}(\tau,u+i).
\end{equation}
Also, one can invert the characteristic functions to get cdf and 
\eqref{eq:heston_discounters} becomes
\begin{equation}
\begin{split}
    P_{j}&=1-F^{*}_{j}(\ln K)=\frac{1}{2\pi}\lim_{y\rightarrow+\infty}
    \lim_{L\rightarrow\infty}\int_{-L}^{L} \frac{e^{-iu\ln K}-e^{-iuy}}{iu}\Phi_{j} du \\
    &=\frac{1}{2\pi}\lim_{y\rightarrow+\infty}\lim_{L\rightarrow\infty}
    \int_{-L}^{L} \frac{1-e^{-iuy}}{iu}\Phi_{j} du+\frac{1}{2\pi}
    \lim_{L\rightarrow\infty}\int_{-L}^{L} \frac{e^{-iu\ln K}-1}{iu}\Phi_{j} du \\
    &=\frac{1}{2}+\frac{1}{\pi}\int_{0}^{\infty} \Re
    \left(\frac{e^{-iu\ln K}\Phi_{j}}{iu}\right) du.
\end{split}
\end{equation}
The third equal sign adds 1 on the numerators so the integrands 
can be finite at $u=0$. The last equal sign is based on characteristic 
function's properties $\Phi(0)=1,\,\Phi(u)=\overline{\Phi(-u)}$ 
and a useful integral 
\begin{equation}
    \int_{0}^{+\infty} \frac{\sin\alpha x}{x} dx=
    \begin{cases}
        \frac{\pi}{2} & \alpha>0 \\
        0 & \alpha=0 \\
        -\frac{\pi}{2} & \alpha<0
    \end{cases}
\end{equation}

Finally, we can obtain the European call option price
\begin{equation}
\label{eq:heston_call_formula}
    C(S,v,t;K,T,r)=P(t,T)\left[\frac{1}{2}(F_{t}-K)+\frac{1}{\pi}
    \int_{0}^{\infty} \left(F_{t}\cdot f_{1}-K\cdot f_{2}\right) du\right]
\end{equation}
where $F_{t}=\frac{S_{t}}{P(t,T)}=S_{t}e^{r(T-t)}$ and 
\begin{equation}
    \begin{split}
        f_{1}&=\Re\left(\frac{e^{-iu\ln K}\varphi(u-i)}{iuF_{t}}\right), \\
        f_{2}&=\Re\left(\frac{e^{-iu\ln K}\varphi(u)}{iu}\right), \\
        \varphi(u)&=\Phi_{2}(\tau;u)=\exp({C(\tau,u)+D(\tau,u)v+iu\ln S_{t}}), \\
        C(\tau,u)&=\frac{\kappa\theta}{\sigma^{2}}\left((\kappa-i\rho\sigma u+d(u))
        \tau-2\ln\left(\frac{1-g(u)e^{d(u)\tau}}{1-g(u)}\right)\right), \\
        D(\tau,u)&=\frac{\kappa-i\rho\sigma u+d(u)}{\sigma^{2}}
        \left(\frac{1-e^{d(u)\tau}}{1-g(u)e^{d(u)\tau}}\right), \\
        d(u)&=\sqrt{(\kappa-i\rho\sigma u)^{2}+\sigma^{2}(u^{2}+iu)}, \\
        g(u)&=\frac{\kappa-i\rho\sigma u+d(u)}{\kappa-i\rho\sigma u-d(u)}.
    \end{split}
\end{equation}

\begin{remark}
There are mainly two troubles when doing numerical quadrature in 
\eqref{eq:heston_call_formula}. One is that the Heston factors 
$f_{1,2}$ are usually oscillatory. The other difficulty is the 
calculations of complex logarithm and square root, which can result 
in numerical instability.\footnote{There are discussions in 
HestonTrap.pdf and NotSoComplexLogarithmsInTheHestonModel.pdf.}
\end{remark}

\begin{remark}
The put option price can be obtained by put-call parity 
\eqref{eq:call_put_parity}.
\end{remark}

\subsection{Finite Difference Schemes}

\subsection{Numerical Integration}
Gauss-Kronrod quadrature

\newpage
\section{Heston-Nandi GARCH Model}
This section introduce Heston's further research in 2000.\footnote{
All reference in this section can be found in Heston2000\_RFS.pdf} 
In their paper, Heston and Nandi develops a closed-form option 
valuation formula for a spot asset whose variance follows a 
Generalized Auto-regressive Conditional Heteroskedasticity (GARCH) 
process that can be correlated with the returns of the spot asset. 
It provides the first readily computed option formula for a random 
volatility model that can be estimated and implemented solely on 
the basis of observables. They also show that the improvement, 
to \textit{ad hoc} BS model, is largely due to the ability 
of the GARCH model to simultaneously capture the correlation of 
volatility with spot returns and the path dependence in volatility.

\subsection{Time Series Model}
Under the physical measure $P$, the discertized form of SDE 
\eqref{eq:Heston_SDE} is a GARCH$(1,1)$ process, which looks 
like Engle and Ng's NGARCH rather than Duan's classic 
GARCH. We assume the variance process follows GARCH$(p,q)$:
\begin{equation}
\label{eq:heston_GARCH}
\begin{split}
    r_{t}&=\ln\left(\frac{S_{t}}{S_{t-1}}\right)
    =r+\lambda v_{t}+\sqrt{v_{t}}z_{t}, \\
    v_{t}&=w+\sum_{i=1}^{p}b_{i}v_{t-i}
    +\sum_{j=1}^{q}a_{j}(z_{t-j}-c_{j}\sqrt{v_{t-j}})^{2} \\
    &=w+\sum_{i=1}^{p}b_{i}v_{t-i}+\sum_{j=1}^{q}a_{j}
    \frac{(r_{t-j}-r-\lambda v_{t-j}-c_{j}v_{t-j})^{2}}{v_{t-j}},
\end{split}
\end{equation}
where $S_{t}$ is the underlying asset price at time $t$, $r$ is 
the continuously compounded risk-free rate, $v_{t}$ is the 
conditional variance of the log return between $t-1$ and $t$ and 
is known at time $t-1$, $\lambda v_{t}$ is \textit{equity risk 
premium}, $z_{t}$ is a standard BM.

At this point we cannot value options or other contingent because 
we do not know the risk-neutral distribution of the spot price. 
Thus, we rearrange SDE \eqref{eq:heston_GARCH} in the form 
\begin{equation}\label{eq:heston_GARCH_emm}
    \begin{split}
        r_{t}&=r-\frac{1}{2}v_{t}+\sqrt{v_{t}}z^{*}_{t}, \\
        v_{t}&=w+\sum_{i=1}^{p}b_{i}v_{t-i}+\sum_{j=2}^{q}a_{j}
        (z_{t-j}-c_{j}\sqrt{v_{t-j}})^{2}+a_{1}(z^{*}_{t-1}-
        c^{*}_{1}\sqrt{v_{t-1}})^{2},
    \end{split}
\end{equation}
where $z^{*}_{t}=z_{t}+(\lambda+1/2)\sqrt{v_{t}}$ and 
$c_{1}^{*}=c_{1}+\lambda+1/2$.

There is no reason for the risk-neutral distribution of $z^{*}_{t}$ 
to be normal because BS price do not follow absence of arbitrage 
with discrete-time trading. In order for $z^{*}_{t}$ to have a 
standard normal risk-neutral distribution, we need a assumption: 
\textit{the value of a call option with one period to expiration 
obeys the Black-Scholes-Rubinstein formula}.

\begin{remark}
Following Heston and Nandi, we will focus on the first-order, i.e. 
$p=q=1$. The first-order process remains stationary with finite 
mean and variance if $b_{1}+a_{1}c_{1}^{2}<1$. In the multiple 
factor case, one must add a condition that the roots of 
$x^{p}-\sum_{i=1}^{p}(b_{i}+a_{i}c_{i}^{2})x^{p-i}$ lie inside the 
unit circle.
\end{remark}

\subsection{Estimation of Model Parameters}
Typically, we use GARCH$(1,1)$, which converges to continuous time 
Heston's SV model, when we deal with the short term data, like 
daily data. Thus, we need to estimate five parameters 
$(a,b,c,w,\lambda)$. We do this with the Maximum Likelihood 
Estimation (MLE) used by Bollerslev (1986) and many others.
\footnote{Q: How to solve? GD or manually?} Since 
$z_{t}\sim N(0,1)$, the likelihood function is 
\begin{equation}
    L(a,b,c,w,\lambda;r_{t})
    =\prod_{t=1}^{T}\frac{1}{\sqrt{2\pi v_{t}}}\exp\left[
        -\frac{(r_{t}-r-\lambda v_{t})^{2}}{2v_{t}}\right].
\end{equation}
Then the log-likelihood function is 
\begin{equation}
    \ln L=-\frac{1}{2}\sum_{t=1}^{T}\left[\ln(2\pi v_{t})
    +\frac{(r_{t}-r-\lambda v_{t})^{2}}{v_{t}}\right],
\end{equation}
where 
\begin{equation}
    v_{t}=w+bv_{t-1}+a
    \frac{(r_{t-1}-r-\lambda v_{t-1}-cv_{t-1})^{2}}{v_{t-1}}.
\end{equation}

\begin{remark}
Following Heston and Nandi, we assume that $v_{1}$ equals to the 
sample variance of $r_t$. Due to strong mean reversion of 
volatility, all results should be insensitive to the starting value 
$v_{1}$.
\end{remark}

\subsection{Call Option Pricing}
There are mainly two methods about pricing the call option after 
we get the parameters in \eqref{eq:heston_GARCH}. The first one 
is Monte Carlo Simulation under EMM, which will be slow and 
computationally intensive for empirical work. In contrast, we can 
use the Heston model's analytic formula introduced in Section 
\ref{sec:Heston}.

Let $f(u)$ be the moment generating function of the log asset price 
under the physical measure. Following the derivation in Section 
\ref{sec:heston_Riccati}, we can get 
\begin{equation}
    f(u)=E_{t}[e^{u\ln S_{T}}]=E_{t}[S_{T}]
    =S_{t}^{u}\exp(A_{t}+B_{t}v_{t+1}).
\end{equation}

The two coefficients can be calculated recursively, by working 
backward from the maturity date and using the terminal conditions
\footnote{The details are in the Appendix A of Heston\_2000.pdf.}
\begin{equation}
    \begin{split}
        A_{t}&=A_{t+1}+ur+B_{t+1}w-\frac{1}{2}\ln (1-2aB_{t+1}), \\
        B_{t}&= u(\lambda+c)-\frac{1}{2}c^{2}+bB_{t+1}
        +\frac{(u-c)^{2}}{2(1-2aB_{t+1})}, \\
        A_{T}&=B_{T}=0.
    \end{split}
\end{equation}

We can rewrite the closed-form formula 
\eqref{eq:heston_call_formula} as followed:
\begin{equation}
\label{eq:heston_call_formula_simplified}
\begin{split}
    C=\frac{1}{2}S_{t}&+\frac{e^{-r(T-t)}}{\pi}
    \int_{0}^{\infty} \Re\left[\frac{K^{-iu}f^{*}(iu+1)}{iu}
    \right] du \\
    &-Ke^{-r(T-t)}\left\{\frac{1}{2}+\frac{1}{\pi}
    \int_{0}^{\infty} \Re\left[\frac{K^{-iu}f^{*}(iu)}{iu}
    \right] du\right\},
\end{split}
\end{equation}
where $f^{*}(iu)$ is the conditional characteristic function of 
the log asset price under the EMM.

\begin{remark}
Observing the GARCH forms in \eqref{eq:heston_GARCH} and 
\eqref{eq:heston_GARCH_emm}, we find the similarity if $p=q=1$. 
Since we have the formula of $f(u)$, the calculation of $f^{*}(iu)$ 
only needs to replace $\lambda$ with $-1/2$ and $c$ with 
$c+\lambda+1/2$.
\end{remark}

\end{document}
